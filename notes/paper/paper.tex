\documentclass[useAMS,usenatbib]{mn2e}
\bibliographystyle{mn2e}
\pdfoutput=1
\pdfminorversion=5

%\usepackage{widetext}
\usepackage{graphicx}
\usepackage{dcolumn}
\usepackage{bm}
\usepackage{amssymb,amsmath,bm}  
\usepackage{color}
\usepackage[colorlinks,linkcolor=red,citecolor=blue,urlcolor=blue ]{hyperref}
\usepackage{multirow}
\usepackage[utf8]{inputenc}
\usepackage{balance}
\usepackage{enumitem}
\usepackage{lipsum}
\newcommand{\nv}{\hat{\bf n}}
\newcommand{\TODO}[1]{{\bf TODO:} #1}
\newcommand{\kalo}{Karhunen-Lo\`{e}ve\,}
\newcommand{\jcap}{JCAP}
\newcommand{\mnras}{MNRAS}
\newcommand{\aap}{A\&A}
\newcommand{\aaps}{A\&AS}
\newcommand{\apjs}{ApJS}
\newcommand{\apj}{ApJ}
\newcommand{\apjl}{ApJL}
\newcommand{\prd}{Phys.~Rev.~D}
\newcommand{\aj}{Astron. Journal}
\newcommand{\pasp}{Publications of the ASP}
\newcommand{\nar}{New Astronomy Review}
\newcommand{\procspie}{Proceedings of the SPIE}
\newcommand{\physrep}{Physics Reports}

\title[Tomographic measurement of the gas pressure through galaxy-tSZ cross-correlations]{Tomographic measurement of the gas pressure through galaxy-tSZ cross-correlations}
\author[David Alonso]{David Alonso$^1$, Maciej Bilicki$^2$, Nick Koukoufilippas$^1$, John A. Peacock$^3$\\
                      $^{1}$Department of Physics, University of Oxford, Keble Road, Oxford, OX1 3RH, UK\\
                      $^{2}$Center for Theoretical Physics, Polish Academy of Sciences, al. Lotnik\'ow 32/46, 02-668, Warsaw, Poland\\
                      $^{3}$Institute for Astronomy, University of Edinburgh, Royal Observatory, Edinburgh EH9 3HJ, United Kingdom
                      }

\begin{document}
  \date{\today}
  \pagerange{1--18} \pubyear{2018}
  \maketitle

\begin{abstract}
  \lipsum[1]
\end{abstract}

\begin{keywords}
  cosmology: large-scale structure of the Universe -- methods: data analysis
\end{keywords}

\section{Introduction}\label{sec:intro}
  \TODO{Motivation:}
  \begin{itemize}
    \item Measure $\langle P_e\rangle(z)$ or $(1-b_H)(z)$.
    \item Conversely, gain info on HOD properties (?)
    \item Pressure-galaxy relation (?)
  \end{itemize}
  \TODO{Define tSZ}

\section{Theory}\label{sec:theory}
  \subsection{Projected fields and angular power spectra}\label{ssec:theory.cls}
    Our study will focus on the cross-correlation of the 2D galaxy overdensity in tomographic redshift bins $\delta_g$ and maps of the tSZ Compton-$y$ parameter. Both of these observables can be described as a projected quantity $u(\nv)$, where $\nv$ is a unit vector on the sphere, related to a three-dimensional field $U({\bf r})$ $V({\bf r})$ through some radial kernel $W_u(\chi)$:
    \begin{equation}
      u(\nv)=\int d\chi\,W_u(\chi)\,U(\chi\nv).
    \end{equation}
    Any projected quantity can be decomposed into its spherical harmonic coefficients $u_{\ell m}$, the covariance of which is the so-called angular power spectrum $\langle u_{\ell m}v^*_{\ell' m'}\rangle\equiv C^{uv}_\ell\delta_{\ell\ell'}\delta_{mm'}$.

    The angular power spectrum can be related to the 3D power spectrum of the associated 3D fields $P_{UV}$ as:
    \begin{equation}
      C_\ell^{uv} = \int d\chi \frac{W_u(z)W_v(z)}{\chi^2(z)}\,P_{UV}\left( z, k=\frac{\ell+1/2}{\chi(z)} \right).
    \end{equation}
    Here, the 3D power spectrum is analogously defined as the variance of the Fourier-space 3D quantities:
    \begin{equation}
      \left\langle U({\bf k})V^*({\bf k}')\right\rangle = (2\pi)^3\,\delta({\bf k}-{\bf k}')\,P_{UV}(k).
    \end{equation}
    We will model the 3D power spectrum using the halo model described in \ref{ssec:theory.hm} below.

    In this formalism, the 3D quantities associated to $\delta_g(\nv)$ and $y(\nv)$ are the 3D galaxy overdensity $\Delta_g({\bf x})=n_g({\bf x})/\bar{n}_g-1$ and the electron pressure $P_e({\bf x})$ respectively, where $n_g$ is the galaxy number density. The associated radial kernels are:
    \begin{equation}
      W_g(\chi)=\frac{H(z)}{c}\,\phi_g(z),\hspace{12pt}W_y(\chi)=\frac{\sigma_T}{m_ec^2}\frac{1}{1+z},
    \end{equation}
    where $\phi_g(z)$ is the normalized galaxy redshift distribution, $H(z)$ is the expansion rate, $\sigma_T$ is the Thomson scattering cross-section, and $m_e$ is the electron mass.

  \subsection{Halo model predictions}\label{ssec:theory.hm}
    The halo model \TODO{cite} describes the spatial fluctuations of any quantity in terms of the contributions to them of all dark matter haloes, under the assumption that all the matter in the Universe is contained in those haloes. We will only quote here the final results regarding the halo model prediction for power spectra, and refer the reader to \TODO{cite} for further details.
    
    Let $U(r|M)$ be the profile of a given quantity as a function of the comoving distance $r$ to the centre of a halo of mass $M$, and let $U(k|M)$ be its Fourier transform:
    \begin{equation}
      U(k|M)\equiv4\pi \int_0^\infty dr\,r^2\,\frac{\sin(kr)}{kr}U(r|M).
    \end{equation}

    The halo model prediction for the cross-power spectrum $P_{UV}$ then consists of two contributions, from the so-called 1-halo term and 2-halo term:
    \begin{equation}
      P_{UV}(k)=P^{1h}_{UV}(k)+P^{2h}_{UV}(k).
    \end{equation}
    Each of these can be estimated in terms of the Fourier-space profiles as:
    \begin{align}
      &P^{1h}_{UV}(k)=\int dM\,\frac{dn}{dM}\,\langle U(k|M)\,V(k|M)\rangle,\\
      &P^{2h}_{UV}(k)=b_U(k)\,b_V(k)\,P_L(k),\\
      &b_U(k)\equiv\int dM\frac{dn}{dM}\,b_h(M)\,\langle U(k|M)\rangle.
    \end{align}
    Here, $P_L(k)$ is the linear matter power spectrum, $dn/dM$ is the halo mass function and $b_h(M)$ is the halo bias \TODO{cites}. %Further details are discussed in Appendix \ref{app:hm}.
    
    \TODO{describe $1h$-$2h$ transition correction.}
    
    \subsubsection{Galaxies and the halo occupation distribution}\label{sssec:theory.hm.hod}
      To model the galaxy overdensity $\Delta_g$ we will use a Halo Occupation Distribution (HOD) model as prescribed by \TODO{cite}. The HOD models the galaxy content of dark matter halos as being made up of central and satellite galaxies. Centrals lie at the center of the halo, while satellites are distributed according to a profile $u_s(r|M)$. Halos can have zero or one central, and the mean number of centrals for a halo of mass $M$ is modelled as a smoothed step function
      \begin{equation}
        \langle N_c(M)\rangle=\frac{1}{2}\left[1+{\rm erf}\left(\frac{\log(M/M_{\rm min})}{\sigma_{\rm lnM}}\right)\right].
      \end{equation}
      We then assume that satellites can only be formed if a halo has a central and has a mass larger than some threshold $M_0$. In that case, average number of satellites is a power law of the form:
      \begin{equation}
        \langle N_s(M)\rangle=\Theta(M-M_0)\,\left(\frac{M-M_0}{M_1'}\right)^{\alpha_s}.
      \end{equation}
      This is the same model used by \TODO{cite}. For simplicity, we will fix $M_0$ to $M_{\rm min}$, $\sigma_{\rm lnM}=0.15$ and $\alpha_s$=1 \TODO{cite}, leaving only two free parameters: $M_{\rm min}$ and $M_1$.
      
      Besides their mean values, we also need to specify the statistics of $N_c$ and $N_s$. Following the standard practice \TODO{cite}, we assume $N_c$ to have a binomial distribution with probability $p=\langle N_c\rangle$, and $N_s$ to be Poisson-distributed.

      Putting everything together, the moments of the galaxy overdensity Fourier profile are \TODO{cite}:
      \begin{align}
        &\langle u_g(k)\rangle=\bar{n}_g^{-1}\langle N_c\rangle\left[1+\langle N_s\rangle\,u_s(k)\right],\\
        &\langle |u_g(k)|^2\rangle=\bar{n}_g^{-2}\langle N_c\rangle\left[\langle N_s\rangle^2u_s^2(k)+2\langle N_s\rangle u_s(k)\right],
      \end{align}
      where the mean number density $\bar{n}_g$ is
      \begin{equation}
        \bar{n}_g\equiv\int dM\,\frac{dn}{dM}\langle N_c\rangle\left[1+\langle N_s\rangle\right],
      \end{equation} 
      and we have suppressed the mass dependence of all quantities for simplicity.

      Finally, we will assume that the satellites follow the matter distribution, and therefore $u_s(k|M)$ is given by a truncated Navarro-Frenk and White profile \TODO{cite}:
      \begin{align}
        u_s(k|M)=&\left[\log(1+c_\Delta)-\frac{c_\Delta}{(1+c_\Delta)}\right]^{-1}\\
               &\left[\cos(q)\left({\rm Ci}((1+c_\Delta)q)-{\rm Ci}(q)\right)\right.\\\nonumber
               &\left.+\sin(q)\left({\rm Si}((1+c_\Delta)q)-{\rm Si}(q)\right)\right.\\&\left.-\sin(c_\Delta q)/(1+c_\Delta q)\right],
      \end{align}
      where $q\equiv kr_\Delta/c_\Delta$, $r_\Delta$ and $c_\Delta$ are the halo radius and concentration defined in Section \ref{sssec:theory.hm.cm}, and $\{{\rm Ci}, {\rm Si}\}$ are the cosine and sine integrals.
      
    \subsubsection{tSZ and pressure profiles}\label{sssec:theory.hm.pe}
      In order to describe the electron pressure in a halo, we use the generalized NFW profile (GNFW) described in \cite{2010A&A...517A..92A} and used in the Planck tSZ cluster analysis \cite{2016A&A...594A..24P}. This profile takes the form:
      \begin{equation}
        P_e(r)=P_*\,p(r/r_{500c}),
      \end{equation}
      where $r_{500c}$ is the cluster radius enclosing an overdensity of 500 times the critical density (see Section \ref{sssec:theory.hm.cm}). The normalization $P_*$ is given by
      \begin{equation}
        P_*=6.41\,\left(1.65 {\rm eV}\,{\rm cm}^{-3}\right)h_{70}^{8/3}
        \left(\frac{h_{70}(1-b_H)M_{500c}}{3\times10^{14}M_\odot}\right)^{0.79},
      \end{equation}
      where $h_{70}=H_0/(70 {\rm km}\,{\rm s}^{-1}\,{\rm Mpc}^{-1}$, and $(1-b_H)$ is the so-called hydrostatic bias parameter. The GNFW form factor is
      \begin{equation}
        p(x)=(c_P x)^{-\gamma}\left[1+(c_P x)^\alpha\right]^{(\gamma-\beta)/\alpha},
      \end{equation}
      with $(\alpha,\beta,\gamma,c_P)=(1.33,4.13,0.31,1.81)$.
      
      \TODO{Blah about $(1-b_H)$}. Other pressure profiles have been proposed in the literature (e.g. \TODO{cite}), but we chose this parametrisation in order to be able to relate our measurement of $(1-b_H)$ to the results of \cite{2016A&A...594A..24P}.
      
      Within the halo model description, and assuming a log-normal $y$-mass relation, the pressure profile cumulants are given by:
      \begin{align}
        &\langle u_y(k|M)\rangle=P_e(k),\\
        &\langle u_y^2(k|M)\rangle=P_e^2(k)\,e^{\sigma_{\ln Y}^2},
      \end{align}
      where $P_e(k)$ is the Fourier transform of the GNFW profile and $\sigma_{\ln Y}=0.173\pm0.023$ is the intrinsic logarithmic scatter in the $y$-mass relation \cite{2016A&A...594A..24P}. Note that, since we do not make use of the tSZ auto-correlation, we will not use the second order cumulant of the pressure profile in our analysis. However, since we analyse the galaxy-tSZ correlation, we need to model the covariance between the galaxy overdensity and pressure profiles. For simplicity, we adopt a simple parametrization:
      \begin{equation}
        \langle u_y(k|M) u_g(k|M)\rangle = (1+\rho_{yg})\langle u_g(k|M)\rangle \langle u_y(k|M)\rangle,
      \end{equation}
      where $\rho_{yg}$ is a free parameter that determines the sign of the correlation between galaxy abundance and pressure. Marginalising over this parameter has the added advantage of reducing the sensitivity of our final constraints on $1-b_H$ on the details of the cross-spectrum model in the 1-halo regime, where both parameters are completely degenerate.              
      
    \subsubsection{Concentration-mass relation and mass definitions}\label{sssec:theory.hm.cm}
      Halo radii $r_\Delta$ are usually defined as the size of the sphere containing a given mass $M_\Delta$:
      \begin{equation}
        M_\Delta = \frac{4\pi}{3}\rho_*(z)\Delta r^3_\Delta.
      \end{equation}
      Common choices for $\rho_*$ are the critical density $\rho_c=3H^2(z)/4\pi G$ and the matter density $\rho_M(z)$. The spherical overdensity parameter $\Delta$ is usually chosen within the range $\sim(200,500)$ and sometimes defined as the quantity yielding the virial radius in the spherical top-hat collapse model $\Delta_v$ \cite{1998ApJ...495...80B} \TODO{more refs?}.

      Ideally we would like to use the same mass definition (i.e. choice of $\Delta$ and $\rho_*$) for the mass function, the mass-concentration relation $c_\Delta(M)$ and the calibrated pressure profile. Unfortunately while the GNFW profile is calibrated to $\Delta_{500c}$ (where $c$ denotes critical density), the mass functions of \cite{2008ApJ...688..709T,2010ApJ...724..878T} are only provided for $\rho_M$-based mass definitions, and the concentration-mass relation of \cite{2008MNRAS.390L..64D} was only estimated for $\Delta=200$ (for critical and matter densities) and for $\Delta=\Delta_v$. To overcome this issue, we follow the procedure used by \cite{2016A&A...594A..24P} and \cite{2018MNRAS.477.4957B}: our baseline mass definition is $\rho_c$-based with $\Delta=500$, as used by \cite{2010A&A...517A..92A}. At each redshift, we translate this into a $\Delta$ value for a $\rho_M$-based definition, which we use to compute the mass function from the parametrizations of \cite{2008ApJ...688..709T,2010ApJ...724..878T}. We also rederive the concentration-mass relation of \cite{2008MNRAS.390L..64D} for a $\rho_c$-based $\Delta=500$ from their $\Delta=200$ parametrization by integrating the NFW profile to the corresponding halo radius. Within the redshifts covered by our analysis we find that this is well fit by:
      \begin{equation}
        c_{500c}(M,z)= A\,(M/M_{\rm pivot})^B\,(1+z)^C,
      \end{equation}
      with $M_{\rm pivot}=2.7\times10^{12}\,M_\odot$ and $(A,B,C)=(3.67,\,-0.0903,\,-0.51)$.

\section{Data}\label{sec:data}
  \subsection{The Compton-$y$ map}\label{ssec:data.y}
    \TODO{Discuss MILCA and NILC maps, as well as associated masks.}
  \subsection{2MPZ and WI$\times$SC}\label{ssec:data.g1}
    \TODO{Discuss the catalogs briefly and note the systematics contamination in WISC}
  \subsection{The SDSS DR12 photometric sample}\label{ssec:data.g2}
    \TODO{Same as above}

\section{Methods}\label{sec:methods}
  \subsection{Estimating power spectra}\label{ssec:methods.cls}
    We measure all auto- and cross-power spectra between the different redshift bins and the $y$ maps through the pseudo-$C_\ell$ approach \TODO{cites} as implemented in \TODO{cite}. Details about the method can be found in \TODO{cite}, but we provide a brief description here for completeness. For an incomplete sky coverage, a given field observed on the sphere, $\tilde{u}(\nv)$, can be modelled as a product of the true underlying field $u$, and a sky mask $w$
    \begin{equation}
      \tilde{u}^{\rm obs}(\nv) = w(\nv)\,u(\nv).
    \end{equation}
    In the simplest scenario, the mask $w$ is simply a binary map ($w=0$ or 1) selecting the pixels in the sky that have been observed. More generally, $w$ can be designed to optimally up- or downweight different regions in an inverse-variance-weighted manner. Through the convolution theorem, the spherical harmonic transform of the observed field is a convolution of the harmonic transforms of the true field and the mask. This then translates into a similar result for the ensemble average of the observed power spectra $\tilde{C}^{uv}_\ell$:
    \begin{equation}
      \tilde{C}^{uv}_\ell = \sum_{\ell'}\,M^{uv}_{\ell \ell'}\, C^{uv}_{\ell'},
    \end{equation}
    where $C^{uv}_\ell$ is the true underlying power spectrum. $M^{uv}_{\ell \ell'}$ is the so-called mode-coupling matrix, which depends solely on the masks of both fields, and which can be computed analytically. The pseudo-$C_\ell$ approach is then based on 


    \TODO{Describe pseudo-$C_\ell$ method.}
  \subsection{Covariance matrices}\label{ssec:methods.cov}
    \TODO{Describe hybrid covariance matrix used (Jackknife $+$ analytic $+$ trispectrum.}
  \subsection{Likelihood}\label{ssec:methods.like}
    \TODO{Describe likelihood, free parameters etc.}

\section{Results}\label{sec:results}
  \subsection{Fiducial results}\label{ssec:results.fid}
    \TODO{Describe fiducial measurement of $(1-b)$ from 2MPZ $+$ WISC, using whatever mass function was used for the Planck clusters analysis.}
    \TODO{Discuss these results: is it compatible with a constant, discuss mass dependence (from SZ source masking), translate into $\langle bP_e \rangle$ and place in context of existing measurements.}

  \subsection{Systematics analysis}\label{ssec:results.syst}
    \subsubsection{Large-scale galaxy clustering systematics}\label{sssec:results.syst.gg}
      \TODO{Describe different efforts to understand/mitigate/remove systematics in WISC (removing first point, fitting for $C_\ell$ template, template deprojection of stars and other stuff).}
      \TODO{Compare with SDSS results if available}
    \subsubsection{Photometric redshift systematics}\label{sssec:results.syst.pz}
      \TODO{Compare results with different $N(z)$ choices.}
    \subsubsection{tSZ contamination}\label{sssec:results.syst.y}
      \TODO{Compare results with NILC map}
      \TODO{Quantify CIB contamination}

\section{Discussion}\label{sec:discussion}
  \lipsum[2]

\section*{Acknowledgements}
  We thank Thor the almighty and the Beach Boys for useful comments and discussions. DA acknowledges support from the Beecroft trust and from the Science and Technology Facilities Council (STFC) through an Ernest Rutherford Fellowship, grant reference ST/P004474/1.
  
\setlength{\bibhang}{2.0em}
\setlength\labelwidth{0.0em}
\bibliography{paper}

\appendix
%\onecolumngrid
\section{An appendix}\label{app:app}
  \lipsum[3]

\end{document}
